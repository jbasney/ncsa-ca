\documentclass[10pt]{article}
\usepackage[margin=1in]{geometry}

\usepackage{times}
\usepackage{graphicx}
\usepackage{url}
\usepackage[pdftex,colorlinks=false]{hyperref}
\usepackage{epsfig}

%\bibliographystyle{abbrv}
\bibliographystyle{plain}
\begin{document}

\title{Certificate Policy and Practice Statement}
\author{National Center for Supercomputing Applications (NCSA)}

\date{Version 1.0 (June 14, 2006)}

\maketitle

\section{INTRODUCTION}

\subsection{Overview}

This Certificate Policy and Practice Statement (herein referred to as
the "Policy") specifies minimum requirements for the issuance and
management of digital certificates that shall be used in
authenticating users accessing National Center for Supercomputing
Application at the University of Illinois (herein referred to as
"NCSA") resources and the resources of other entities (relying
parties) which accept those certificates. The Policy is issued and
administered under the authority of the NCSA Policy Management
Authority (herein referred to as the "PMA"; see Section 1.4.2 for
contact details).

This document is based upon the framework provided by the Global Grid
Forum Certificate Policy Model (draft-gridforum-CP.txt v.6, September,
2001 - http://caops.es.net/Documents/Draft-GGF-CP-06.pdf).  The
structure and content of that document was derived from the Internet
Engineering Task Force RFC 3647 (Internet X.509 Public Key
Infrastructure Certificate Policy and Certification Practices
Framework, http://www.ietf.org/rfc/rfc3647.txt). The Fermilab CPS was
invaluable in crafting this document.

The NCSA PKI infrastructure includes two certificate-signing
authorities (CAs). One is a traditional CA that issues long-lived
certificates to hosts, services and users requiring long-lived
certificates. This CA is henceforth referred to as the “NCSA-CA”. This
document describes the policy for the NCSA-CA.

The other certificate-signing authority issues only short-lived
credentials to users and is henceforth referred to as the “NCSA
Short-lived Certificate Service” or “NCSA-SLCS”.

This document covers the policy that applies to the NCSA-CA.

\subsection{Document name and identification}

Document title: Certificate Policy and Practice Statement for the NCSA
CA

This Policy is published at
http://www.ncsa.uiuc.edu/UserInfo/Security/policy/CA/

Document version: v1.0-DRAFT

Document date: XXX

OID: 1.3.6.1.4.1.4670.100.1.1
[SLCS: 1.3.6.1.4.1.4670.100.2.1]

\subsection{PKI participants}

\subsubsection{Certification authorities}

This policy is valid for the NCSA-CA. The NCSA-CA will only sign end
entity certificates. There are no subordinate CAs.

\subsubsection{Registration authorities}

NCSA maintains a user database of legitimate users of NCSA
computational systems. This user database is exposed through a
Kerberos domain, which serves the Registration Authority role and is
used by the NCSA-CA to authenticate certificate requests.

 The same user database is used as an authentication service for
 NCSA’s users and staff to access NCSA high-performance computing
 resources, NCSA’s email services and other production services.

In the case of user certificates, the NCSA-CA will automatically issue
user certificates with a distinguished name based on the user's
authenticated identity.

In the case of server and service certificate requests, the NCSA-CA
system’s administrators will perform the RA function by validating the
that the user making the request represents the server or service in
question.


\subsubsection{Subscribers}

The certificate authority operating under this Policy will serve the
needs of the NCSA community by providing individual computer account
holders at NCSA with x509v3 digital certificates.  These certificates
may be used for the purpose of authentication, encryption, and digital
signing by those individuals to whom the certificates have been
issued.  The CA may also issue server and service certificates to
computers operating within the NCSA administrative domain and to
computer systems from other domains that are providing services in
support of NCSA projects.

The end entities to be certified in accordance with this policy are: a
natural person or a computer entity (a computer or application).

Persons will be users of NCSA’s computational facilities or staff
employed at NCSA.

Computer entities will be servers or services at NCSA, or at NCSA
partners that provide services to NCSA users and/or staff.

\subsubsection{Relying parties}

NCSA places no restrictions on who may accept certificates it issues.

\subsubsection{Other participants}

No stipulations.

\subsection{Certificate usage}

\subsubsection{Appropriate certificate uses}

One of the purposes of this policy is to promote a wide use of
public-key certificates in many different applications.  These
applications may include, but are not limited to, login
authentication, job submission authentication, encrypted e-mail, and
SSL/TLS encryption for applications capable of making use of these
technologies.

\subsubsection{Prohibited certificate uses}

Other uses of NCSA-CA certificates are not prohibited, but neither are
they supported.

\subsection{Policy administration}

\subsubsection{Organization administering the document}

This policy is administered by:

The National Center for Supercomputing Applications
at the University of Illinois
1205 W. Clark, Urbana IL 61801

\subsubsection{Contact person}

The point of contact for this Policy and other matters related to the
NCSA-CA is the Head of Security Operations for NCSA. Currently this is
Jim Barlow.

James J. Barlow
Head of Security Operations and Incident Response
Phone number: 217-244-6403
Postal address: 1205 W. Clark, Urbana IL 61801
E-mail address: jbarlow@ncsa.uiuc.edu

After hours contact information:

NCSA Security Operations and Incident Response: security@ncsa.uiuc.edu

NCSA 24x7 Operations: 217-244-0710

\subsubsection{Person determining CPS suitability for the policy}

The Head of Security Operations for NCSA leads the PMA for the CA and
is ultimately responsible for determining the suitability of the CPS.

\subsubsection{CPS approval procedures}

As determined by the Head of Security Operations for NCSA.

\subsection{Definitions and acronyms}

Certification Authority (CA) - An authority trusted by 
one or more users to create and assign public key 
certificates. Optionally the CA may create the user's 
keys. It is important to note that the CA is responsible 
for the public key certificates during their whole 
lifetime, not just for issuing them. 
 
CA-certificate - A certificate for one CA's public key
issued by another CA or self signed.  
 
Certificate policy (CP) - A named set of rules that 
indicates the applicability of a certificate to a 
particular community and/or class of application with 
common security requirements. For example, a particular 
certificate policy might indicate applicability of a 
type of certificate to the authentication of electronic 
data interchange transactions for the trading of goods 
within a given price range. 
 
Certification path - An ordered sequence of certificates 
which, together with the public key of the initial 
object in the path, can be processed to obtain that of 
the final object in the path. 

Certification Practice Statement (CPS) - A statement of the practices, 
which a certification authority employs in issuing certificates. 
 
Certificate revocation list (CRL) - A CRL is a time 
stamped list identifying revoked certificates, which is 
signed by a CA and made freely available in a public 
repository. 

Issuing certification authority (issuing CA) - In the context of a 
particular certificate, the issuing CA is the CA that issued the 
certificate (see also Subject certification authority). 
 
Public Key Certificate (PKC) - A data structure 
containing the public key of an end entity and some 
other information, which is digitally signed with the 
private key of the CA which issued it. 
 
Public Key Infrastructure (PKI) - The set of hardware, 
software, people, policies and procedures needed to 
create, manage, store, distribute, and revoke PKCs based 
on public-key cryptography. 
  
Registration authority (RA) - An entity that is 
responsible for identification and authentication of 
certificate subjects, but that does not sign or issue 
certificates (i.e., an RA is delegated certain tasks on 
behalf of a CA). [Note: The term Local Registration 
Authority (LRA) is used elsewhere for the same concept.] 
 
Relying party - A recipient of a certificate who acts in 
reliance on that certificate and/or digital signatures 
verified using that certificate. In this document, the 
terms "certificate user" and "relying party" are used 
interchangeably. 
 
Subject certification authority (subject CA) - In the 
context of a particular CA-certificate, the subject CA 
is the CA whose public key is certified in the certificate. 
 
IPR - Intellectual Property Rights 
 
\section{PUBLICATION AND REPOSITORY RESPONSIBILITIES}

\subsection{Repositories}

The NCSA PKI will maintain a repository at:

http://security.ncsa.uiuc.edu/CA

This repository will contain:

* Self-signed, PEM-formatted certificates for all CAs in the NCSA PKI

* PEM-formatted CRLs for the NCSA PKI

* General information about the NCSA PKI

* The most recent copies of all Certificate Policies for the NCSA PKI
CAs

\subsection{Publication of certification information}

Certificates will be published as they are issued.

\subsection{Time or frequency of publication}

Certificates will be published as they are issued.

The CRL will be published immediately after a certificate has been
revoked as well as on a daily basis.

The Policy shall be published immediately following any update.

\subsection{Access controls on repositories}

There are no restrictions on access to the repositories.

Best effort will be provided to maintain their availability 24x7.

\section{IDENTIFICATION AND AUTHENTICATION}
\subsection{Naming}
\subsubsection{Types of names}

Subject distinguished names are X.500 names, with components varying
depending on the type of certificate.

\subsubsection{Need for names to be meaningful}

The CN component of the subject name in user certificates has no
semantic significance, but should have a reasonable association with
the name of the user. The CN component of the subject name in service
certificates includes the fully qualified DNS name of the service,
which is usually that of the host supporting the service. The
structure of a service’s CN is designed to support SSL, TLS and Globus
services.

\subsubsection{Anonymity or pseudonymity of subscribers}

Anonymity and pseudonymity are not supported.

\subsubsection{Rules for interpreting various name forms}

All subject distinguished names in certificates issued by the NCSA PKI
begin with ‘‘C=US, O=National Center for Supercomputing
Applications’’. The next component will be one of:

OU=Certificate Authorities 

for a CA’s certificate. A CN component will follow the OU, naming the
CA.

OU=Services 

for a service (including a host) certificate issued by the NCSA-CA. A
CN component will follow the OU, naming the service and the fully
qualified domain name (FQDN) at which the service can be contacted,
separated by a slash character. When the service is https, the service
name and separator will be absent, and there may be multiple FQDNs
expressed as a sequence of CN components and/or as a regular
expression in a single CN component.

CN=<User Name>

for a user’s certificate issued by the NCSA-CA. A CN component will
follow containing the user’s full name and, if needed, a numeric value
to disambiguate the name from other users with the same name.

\subsubsection{Uniqueness of names}

Each subject name issued by the NCSA PKI will be issued to one and
only one individual. The NCSA-CA and NCSA-SLCS may issue certificates
with identical names, but only to the same individual.

All names will be prefixed with the relative DN form of ‘‘C=US,
O=National Center for Supercomputing Applications’’ to provide a
globally unique namespace.

For user certificates, a unique "common name" is assigned to each
user, which is based on their legal name. Conflicts with legal names
are resolved by assigning a number to the name to create a unique
name. This common name is used to create unique distinguished names
used in certificates issued by the NCSA-CA to users.

For services certificates, the combination of service name and fully
qualified domain name of the host on which the service resides serves
to disambiguate the name.

\subsubsection{Recognition, authentication, and role of trademarks}

No stipulation.

\subsection{Initial identity validation}

\subsubsection{Method to prove possession of private key}

No stipulation.

\subsubsection{Authentication of organization identity}

NCSA users are identified by their presence in the NCSA user database.

\subsubsection{Authentication of individual identity}

User identity will be authenticated through Kerberos 5
credentials. Requests for service certificates must come from a valid
NCSA User and will be checked against registered system administrator
information.

\subsubsection{Non-verified subscriber information}

Subscribe name and postal address are verified by NCSA's account
creation process. Other gathered information is not verified.

\subsubsection{Validation of authority}

Users making requests for service certificates are verified to be a
validate system administrator for the host in question using NCSA's
internal user data base.

\subsubsection{Criteria for interoperation}

The NCSA PKI is intended to interoperate with other CAs within
TeraGrid and the International Grid Trust Federation.

\subsection{Identification and authentication for re-key requests}

\subsubsection{Identification and authentication for routine re-key}

Every user certificate request is treated as an initial
registration. Subsequent Service and CA certificate requests also
follow the same respective validation steps as initial requests.

\subsubsection{Identification and authentication for re-key after revocation}

If the compromise was limited to just the private key, the request for
re-key will be treated as an initial registration.

If the compromise involved a user's password from the NCSA database,
that password will be reset based on a postal mail to the user using
their known postal address.

\subsection{Identification and authentication for revocation request}

Requests for revocation of service certificates from NCSA computer
security personnel and from administrators of the systems hosting the
services in question will be honored.

CA Certificates will only be revoked at the instigation of NCSA
Operational Security personnel.

Users may request revocation by contacting NCSA Security Operations
and will be identified as deemed appropriate by NCSA Security
Operations.

\section{CERTIFICATE LIFE-CYCLE OPERATIONAL REQUIREMENTS}

\subsection{Certificate Application}

\subsubsection{Who can submit a certificate application}

Any user who appears in NCSA's User Database may request a
certificate.

\subsubsection{Enrollment process and responsibilities}

To receive an entry in NCSA’s user database, a user must be satisfy
one of the following conditions:

-Be a NCSA employee;

-Be a Principal Investigator (PI) with a allocation on NCSA
computational resources approved through an NSF-approved peer review
process (e.g. NRAC);

-Have a "project account" requested on their behalf by an existing PI
using that PI’s allocation;

-Have a "guest Account" requested by NCSA management for key
collaborators of NCSA.

All vetting is done either in person (in the case of employees), by
peer review allocation (in the case of PIs), or by direct personal
contact of a PI or NCSA staff member (in the case of project or guest
accounts).

All initial user passwords are distributed by US postal mail to a
verified address. If a user requires a new password (e.g. they have
lost their password), it will only be distributed via U.S. postal mail
to their known address.

Each user is assigned a unique username used as their Kerberos
principal and Unix login name.

\subsection{Certificate application processing}

\subsubsection{Performing identification and authentication functions}

The NCSA-CA authenticates all certificate requests using the NCSA
Kerberos 5 domain.

\subsubsection{Approval or rejection of certificate applications}

Certificate applications will be approved if the applicant can be
authenticated and, in the case of service certificates, is validated
as an authorized system administrator for the host in question.

\subsubsection{Time to process certificate applications}

Certificate applications are processed with best effort. It is
expected that certificates will be issued within one business day.

\subsection{Certificate issuance}

\subsubsection{CA actions during certificate issuance}

The application is processed and approved as described in the previous
section.

\subsubsection{Notification to subscriber by the CA of issuance of certificate}

User certificates are returned directly to the user through the
application they are using to apply for a certificate.

For other certificates, users will be contacted by email.

\subsection{Certificate acceptance}

\subsubsection{Conduct constituting certificate acceptance}

Certificate acceptance is assumed.

\subsubsection{Publication of the certificate by the CA}

Certificates will be published as they are issued at
http://security.ncsa.uiuc.edu/CA

\subsubsection{Notification of certificate issuance by the CA to other entities}

No notifications to other entities will be performed.

\subsection{Key pair and certificate usage}

\subsubsection{Subscriber private key and certificate usage}

Subscribers must:

* Exercise all reasonable care in protecting the
private keys corresponding to their certificates, including but not
limited to never storing them on a networked file system or otherwise
transmitting them over a network.

* Ensure that the private keys corresponding to their issued service
certificates are stored in a manner that minimizes the risk of
exposure. 

* Observe restrictions on private key and certificate use. 

* Promptly notify the CA operators of any incident involving a
possibility of exposure of a private key. 

\subsubsection{Relying party public key and certificate usage}

Relying parties must 

* Be cognizant of the provisions of this document. 

* Verify any self-signed certificates to their own satisfaction using
out-of-band means. 

* Accept responsibility for checking any relevant CRLs before
accepting the validity of a certificate. 

* Observe restrictions on private key and certificate use. 

* Not presume any authorization of an end entity based on possession
of a certificate from the NCSA PKI or its corresponding private key. 


\subsection{Certificate renewal}

Certificates in the NCSA PKI are not explicitly renewed. Instead the
original subscriber may request a new certificate, using the normal
certificate issuance process.

\subsection{Certificate re-key}

Certificates in the NCSA PKI are not explicitly re-keyed. Instead the
original subscriber may request a new certificate, using the normal
certificate issuance process.

\subsection{Certificate modification}

Certificates in the NCSA PKI are not modified. Instead new
certificates will be issued using the normal certificate issuance
process.

\subsection{Certificate revocation and suspension}

Certificates issued by the NCSA PKI will not be suspended.

\subsubsection{Circumstances for revocation}

User certificates, because of their short lifetimes, will not normally
revoked. Service and CA certificates will be revoked in any of the
following circumstances.

* The private key is suspected or reported to be lost or exposed. 

* The information in the certificate is believed to be, or to have
become inaccurate. 

* The certificate is reported to no longer be needed. 

\subsubsection{Who can request revocation}

The original subscriber for a certificate may request its revocation.

NCSA Security Operations personnel may request revocation of any
certificate issued by the NCSA PKI. Entities other than the subscriber
who suspect a certificate issued by the NCSA PKI may be compromised
should contact NCSA Security Operations.

\subsubsection{Procedure for revocation request}

Requests for revocation should be made by email to
security@ncsa.uiuc.edu or by phone to NCSA Operations 217-244-0710.

\subsubsection{Revocation request grace period}

No constraints.

\subsubsection{Time within which CA must process the revocation request}

Revocation requests will be processed with best effort as quickly as
possible.

\subsubsection{Revocation checking requirement for relying parties}

Relying parties are advised to obtain and consult a valid CRL from
http://security.ncsa.uiuc.edu/CA

\subsubsection{CRL issuance frequency (if applicable)}

CRLs are issued when a certificate is revoked.

CRLs are issued daily.

\subsubsection{Maximum latency for CRLs (if applicable)}

One day.

\subsubsection{On-line revocation/status checking availability}

Aside from the published CRL, no on-line certificate status checking
is available.

\subsubsection{Other forms of revocation advertisements available}

None.

\subsubsection{Special requirements re key compromise}

None.

\subsection{Certificate status services}

Aside from the published CRL, no on-line certificate status checking
is available.

\subsection{End of subscription}

Subscribes may end their subscription by requesting revocation of
their certificate.

No key escrow is performed.

\section{FACILITY, MANAGEMENT, AND OPERATIONAL CONTROLS (11)}

\subsection{Physical controls}

\subsubsection{Site location and construction}

The NCSA PKI will be operated from a computer located with the NCSA
machine room. A identical machine located in the same machine room
serves as a hot spare.

\subsubsection{Physical access}

The NCSA machine room housing the NCSA PKI is staffed 24x7 and access
limited to personnel cleared by NCSA management.

\subsubsection{Power and air conditioning}

No stipulation.

\subsubsection{Water exposures}

No stipulation.

\subsubsection{Fire prevention and protection}

No stipulation.

\subsubsection{Media storage}

No stipulation.

\subsubsection{Waste disposal}

No stipulation.

\subsubsection{Off-site backup}

No stipulation.

\subsection{Procedural controls}

All persons with access to the systems hosting the NCSA PKI will  be
full-time NCSA employees. Personnel will be NCSA Operations staff,
NCSA Security Operations staff, and NCSA System administration staff. 

 When any person with access to the NCSA
systems leaves NCSA or their administrative role, their access will be
revoked and any relevant passwords changed.

\subsubsection{Trusted roles}

No stipulation.

\subsubsection{Number of persons required per task}

No stipulation.

\subsubsection{Identification and authentication for each role}

No stipulation.

\subsubsection{Roles requiring separation of duties}

No stipulation.

\subsection{Personnel controls}

\subsubsection{Qualifications, experience, and clearance requirements}

Operators of the NCSA PKI will be qualified system administrators and
operators at NCSA.

\subsubsection{Background check procedures}

No stipulation.

\subsubsection{Training requirements}

No stipulation.

\subsubsection{Retraining frequency and requirements}

No stipulation.

\subsubsection{Job rotation frequency and sequence}

No stipulation.

\subsubsection{Sanctions for unauthorized actions}

No stipulation.

\subsubsection{Independent contractor requirements}

No stipulation.

\subsubsection{Documentation supplied to personnel}

No stipulation.

\subsection{Audit logging procedures}

\subsubsection{Types of events recorded}

The following items will be logged:

* Certificate requests

* Certificate issuance

* Certificate revocations

* Issued CRLs

* Attempted and successful accesses to the systems hosting the NCSA
PKI

\subsubsection{Frequency of processing log}

No stipulation.

\subsubsection{Retention period for audit log}

Audit logs are maintain indefinitely on NCSA's mass storage system.
(XXX Verify this with Barlow)

\subsubsection{Protection of audit log}

No stipulation.

\subsubsection{Audit log backup procedures}

No stipulation.

\subsubsection{Audit collection system (internal vs. external)}

No stipulation.

\subsubsection{Notification to event-causing subject}

No stipulation.

\subsubsection{Vulnerability assessments}

No stipulation.

\subsection{Records archival}

No records are archived.

\subsection{Key changeover}

The community of known relying parties will be notified of any new CA
public key and it may then be obtained in the same manner as the
previous CA certificates.

\subsection{Compromise and disaster recovery}

\subsubsection{Incident and compromise handling procedures}

All incidents will be handled by NCSA Security Operation and Incident
Response.

\subsubsection{Computing resources, software, and/or data are corrupted}

No stipulation.

\subsubsection{Entity private key compromise procedures}

Any private key compromise will result in revocation of the associated
certificate.

\subsubsection{Business continuity capabilities after a disaster}

No stipulation.

\subsection{CA or RA termination}

No stipulation.

\section{TECHNICAL SECURITY CONTROLS}

\subsection{Key pair generation and installation}
\subsubsection{Key pair generation}

The NCSA PKI does not generate any private keys but its own.

User private keys will be generated by client software on the host
where they will be stored.  They will be stored on non-networked
filesystems.

[FOR SLCS]
 They will normally be stored in the clear, but the
lifetimes of the associated public-key certificates is limited to the
lifetime of the Kerberos credentials used to obtain them, which is
currently no more than 26 hours.
[END SLCS]

System and service administrators will generate private keys for their
services, on the service hosts themselves if at all possible.


\subsubsection{Private key delivery to subscriber}

Not necessary.

\subsubsection{Public key delivery to certificate issuer}

Public keys are delivered under Kerberos authentication and integrity
protection.

\subsubsection{CA public key delivery to relying parties}

The public keys of NCSA PKI CAs are available at
http://security.ncsa.uiuc.edu

\subsubsection{Key sizes}

Public RSA keys shorter than 1024 bits will not be signed.

\subsubsection{Public key parameters generation and quality checking}

No stipulation.

\subsubsection{Key usage purposes (as per X.509 v3 key usage field)}

The NCSA PKI does not enforce key usage restrictions by any means
beyond the X.509v3 extensions in the certificates it issues. In User
and Service certificates, those extensions will mark the associated
keys as valid for Digital Signature and Key Encipherment. CA
certificates will have the Key Usage extension set to allow Digital
Signature, Certificate Signing, and CRL Signing.

\subsection{Private Key Protection and Cryptographic Module Engineering Controls}
\subsubsection{Cryptographic module standards and controls}

The NCSA PKI will use FIPS 140-2 level 3 Hardware Security Modules or
equivalent for storage of all CA private keys.

\subsubsection{Private key (n out of m) multi-person control}

There is no multi-person control of the private key.

\subsubsection{Private key escrow}

CA private keys are not escrowed.

\subsubsection{Private key backup}

CA private keys are replicated on two identical cryptographic modules
on two identical hosts in the NCSA machine room to provide for failure
protection.

\subsubsection{Private key archival}

CA private keys are not archived.

\subsubsection{Private key transfer into or from a cryptographic module}

The CA private keys will initially be replicated on two identical
cryptographic storage modules in a secure manner. After that point
they will not be exported from the cryptographic modules.

\subsubsection{Private key storage on cryptographic module}

CA private keys are stored on cryptographic modules meeting XXX.

\subsubsection{Method of activating private key}

No stipulation.

\subsubsection{Method of deactivating private key}

No stipulation.

\subsubsection{Method of destroying private key}

No stipulation.

\subsubsection{Cryptographic Module Rating}

XXX

\subsection{Other aspects of key pair management}

\subsubsection{Public key archival}

No stipulation.

\subsubsection{Certificate operational periods and key pair usage periods}

NCSA PKI CA certificates will have a lifetime of 10 years.

NCSA-CA User and Service certificates will have a lifetime of not more
than one year.

\subsection{Activation data}
\subsubsection{Activation data generation and installation}

No stipulation.

\subsubsection{Activation data protection}

No stipulation.

\subsubsection{Other aspects of activation data}

No stipulation.

\subsection{Computer security controls}
\subsubsection{Specific computer security technical requirements}

The machines operating the NCSA-PKI run no extraneous network services
and are kept current with respect to relevant security patches. Login
access is subject to hardware-based one-time password authentication
and permitted only for ‘‘administrative’’ personnel that require
access to the system for its operation.

\subsubsection{Computer security rating}

No stipulation.

\subsection{Life cycle technical controls}
\subsubsection{System development controls}

No stipulation.

\subsubsection{Security management controls}

No stipulation.

\subsubsection{Life cycle security controls}

No stipulation.

\subsection{Network security controls}

Other than the access needed to request and obtain certificates, no
network access to the hosts housing the NCSA PKI will be permitted
from outside of NCSA's network.

\subsection{Time-stamping}

No stipulation.

\section{CERTIFICATE, CRL, AND OCSP PROFILES}
\subsection{Certificate profile}
\subsubsection{Version number(s)}

XXX

\subsubsection{Certificate extensions}


For user and service certificates:
   
Basic Constraints (critical): 
CA:false 

X509v3 Subject Key Identifier 
... 

X509v3 Authority Key Identifier 
... 

Key Usage (critical): 
Digital Signature, Key Encipherment 

Netscape Cert Type: 
SSL Client, SSL Server, Object Signing 

Netscape CA Policy URL: 
http://security.ncsa.uiuc.edu/CA/NCSA-CA-Policy.pdf

CRLDistributionPoints:
http://security.ncsa.uiuc.edu/CA/XXX

Additionally, for service certificates:

SubjectAltName:
XXX Fix this
The NCSA Kerberos principal name of the subscriber responsible for the
certificate. The fully qualified domain name will be included as a dnsName .

\subsubsection{Algorithm object identifiers}

XXX not sure what this is

\subsubsection{Name forms}

All certificates will have one of the following name forms:

C=US, O=National Center for Supercomputing Applications, OU=Services,
              CN=<svcname>/<fqdn>
C=US, O=National Center for Supercomputing Applications, OU=Services,
              CN=<fqdn>
C=US, O=National Center for Supercomputing Applications, CN=<user name>
C=US, O=National Center for Supercomputing Applications,
              OU=Certificate Authorities, CN=<ca name>

Where:

  <svcname> is ``host'' or an identifier of the service the
  certificate is associated with.

 <fqdn> is the fully qualified domain name of the host on which the
 service the certificate is associated with is homed.

 <user name> is a unique name for the subscriber, which may have
 appended digits to disambiguate.

 <ca name> is the name of a CA.

\subsubsection{Name constraints}

All certificates issued by the NCSA PKI will have names with the
following prefix:

``C=US, O=National Center for Supercomputing Applications''

\subsubsection{Certificate policy object identifier}

1.3.6.1.4.1.4670.100.1.1
[SLCS: 1.3.6.1.4.1.4670.100.2.1]

\subsubsection{Usage of Policy Constraints extension}

No stipulation.

\subsubsection{Policy qualifiers syntax and semantics}

No stipulation.

\subsubsection{Processing semantics for the critical Certificate Policies extension}

No critical extensions are in use at this time.

\subsection{CRL profile}
\subsubsection{Version number(s)}

The CRL is in vertsion 1 format.

\subsubsection{CRL and CRL entry extensions}

No stipulation.

\subsection{OCSP profile}

OCSP is not supported.

\section{COMPLIANCE AUDIT AND OTHER ASSESSMENTS}

The NCSA PKI will not be audited by an outside party. Certifying,
cross-certifying, and relying organizations may request a review of
NCSA PKI operation.

\subsection{Frequency or circumstances of assessment}

No stipulation.

\subsection{Identity/qualifications of assessor}

No stipulation.

\subsection{Assessor's relationship to assessed entity}

No stipulation.

\subsection{Topics covered by assessment}

No stipulation.

\subsection{Actions taken as a result of deficiency}

No stipulation.

\subsection{Communication of results}

No stipulation.

\section{OTHER BUSINESS AND LEGAL MATTERS}
\subsection{Fees}

No fees will be charged by the NCSA PKI nor any refunds given.

\subsection{Financial responsibility}

No financial responsibility is accepted. 

\subsubsection{Insurance coverage}

No stipulation.

\subsubsection{Other assets}

No stipulation.

\subsubsection{Insurance or warranty coverage for end-entities}

No stipulation.

\subsection{Confidentiality of business information}
\subsubsection{Scope of confidential information}

No stipulation.

\subsubsection{Information not within the scope of confidential information}

No stipulation.

\subsubsection{Responsibility to protect confidential information}

No stipulation.

\subsection{Privacy of personal information}

\subsubsection{Privacy plan}

No stipulation.

\subsubsection{Information treated as private}

No private information is collected by the NCSA PKI.

\subsubsection{Information not deemed private}

No private information is collected by the NCSA PKI.

\subsubsection{Responsibility to protect private information}

No stipulation.

\subsubsection{Notice and consent to use private information}

No stipulation.

\subsubsection{Disclosure pursuant to judicial or administrative
  process}

No stipulation.

\subsubsection{Other information disclosure circumstances}

No stipulation.

\subsection{Intellectual property rights}

The NCSA PKI asserts no ownership rights in certificates issued to
subscribers. 

Acknowledgment is hereby given to the Fermilab PKI, the DOE Science
Grid and to the CERN Certification Authority for inspiration of parts
of this document.

\subsection{Representations and warranties}

The NCSA PKI and its agents make no guarantee about the security or
suitability of a service that is identified by a NCSA certificate. The
certification service is run with a reasonable level of security, but
it is provided on a best effort only basis. It does not warrant its
procedures and it will take no responsibility for problems arising
from its operation, or for the use made of the certificates it
provides.

\subsection{Disclaimers of warranties}

The NCSA PKI denies any financial or any other kind of responsibility
for damages or impairments resulting from its operation.

\subsection{Limitations of liability}

The NCSA PKI is operated substantially in accordance with Fermilab’s
own risk analysis. No liability, explicit or implicit, is accepted.

\subsection{Indemnities}

No stipulation.

\subsection{Term and termination}
\subsubsection{Term}

This policy becomes effective on its posting to
http://security.ncsa.uiuc.edu/CA

\subsubsection{Termination}

This policy may be terminated at any time without warning.

\subsubsection{Effect of termination and survival}

No stipulation.

\subsection{Individual notices and communications with participants}

No stipulation.

\subsection{Amendments}
\subsubsection{Procedure for amendment}

Changes to this document will be presented to the TAGPMA for approval
before taking effect.

Changes will go into effect on the publishing of this document to
http://security.ncsa.uiuc.edu/CA/

\subsubsection{Notification mechanism and period}

Best effort notification of all relying parties will be made with as
much advance notice as possible.

\subsubsection{Circumstances under which OID must be changed}

Any substantial change of policy will incur a change of OID.

\subsection{Dispute resolution provisions}

NCSA Security Operations will resolve all disputes regarding this
policy.

\subsection{Governing law}

Interpretation of this policy is according to the laws of the United
States of America and the State of Illinois, where the conforming CA
is established.

\subsection{Compliance with applicable law}

No stipulation.

\subsection{Miscellaneous provisions}
\subsubsection{Entire agreement}

No stipulation.

\subsubsection{Assignment}

No stipulation.

\subsubsection{Severability}

No stipulation.

\subsubsection{Enforcement (attorneys' fees and waiver of rights)}

No stipulation.

\subsubsection{Force Majeure}

No stipulation.

\subsection{Other provisions}

No stipulation.

\section{COMPLIANCE WITH PROFILE FOR TRADITIONAL X.509 PUBLIC KEY  CERTIFICATION AUTHORITIES}

This document is compliant with the Profile for Traditional X.509
Public Key Certificate Authorities with secured infrastructure Version
4.0 (henceforth referred to as the ``Traditional CA Profile'') . This
section enumerates the requirements of the Traditional CA Profile and
how this policy meets or exceeds those requirements. Each subsection
of this section corresponds that matching secion in the Traditional CA
Profile.

\subsection{About this document}

No requirements specified.

\subsection{General Architecture}

NCSA serves an international user base and qualifies as an
international organization.

NCSA is committing to operating this PKI for the foreseeable future.

\subsection{Identity}

The naming scheme described in section 3.1 ensures that a name will
only ever refer to one entity.

\subsubsection{Identity vetting rules}

Section 4.1 describes NCSA process for vetting its users based on
employment at NCSA, peer review by NSF allocation processes, or direct
contact with NCSA management or PIs.

Section 1.3.2 describes how all certificate requests are authenticated
using NCSA's Kerberos 5 domain.

Section 5.4 describes NCSA's audit procedures.

Section 6.1.3 describes NCSA's use of Kerberos authentication and
integrity protection to deliver public keys to the CA.

\subsubsection{Remove of an authority from the authentication profile accreditation}

No requirement specified.

\subsection{Operational Requirements}

Section 5.1 describes how the hosts operating NCSA's PKI are housed in
NCSA's access-controlled machine room.

Section 6.2 describes NCSA's use of a FIPS 103-2 level 3 Hardware
Security Module to secure the private keys of its CA.

Section 6.5 describes the computer security controls used to limited
network access to the NCSA PKI.

Section 6.3.2 provides lifetime information on certificates in the
NCSA PKI.

NCSA's use of a Hardware Security Module replaces the use of a pass
phrase on the CA private key.

\subsubsection{Certificate Policy and Practice Statement Identification}

Section 1.2 provides an identifying OID for this policy.

Section 9.12 describes the change process for this policy, including
re-accredidation and change of OID.

\subsubsection{Certificate and CRL profile}

Section 7.2 provides the certificate profile for certificates issued
by the NCSA PKI.

Section 6.1 describes the procedures for private key generation and
delivery to the CA under Kerberos 5 protection. 

Section 6.1.5 states that keys must be at least 1024 bits long.

Section 6.3.2 states that end entity certificates will have a lifetime
of not more than one year.

Section 3.1 provides the scheme used by the NCSA PKI to generate
unique common names using representation of actual end entity names.

Section 7.2 provides the profile for the CRL generated by the NCSA
PKI.

\subsubsection{Revocation}

Section 2.3 describes the frequency of publication the NCSA-CA CRL.

Section 4.9.2 describe who may request revocation.

\subsubsection{CA key changeover}

CA key changeover is not discussed in this policy.

XXX Should it be? I don't see where.

\subsection{Site Security}

Section 6.2 describes NCSA's use of a FIPS 103-2 level 3 Hardware
Security Module to secure the private keys of its CA.

\subsection {Publication and Repository responsibilities}

Section 2 provides information a URL for an online repository
maintained by NCSA for the NCSA PKI, intended to be continuously
available.

Section 1.5.2 provides both electronic and physical addresses at NCSA
to contact regarding the NCSA PKI.

This policy and all information in the NCSA PKI repository may be
freely distributed.

\subsection{Audits}

Section 5.4 describes NCSA's audit procedures.

Section 5.2 provides the list of NCSA employees with access to the
NCSA PKI.

\subsection{Privacy and confidentiality}

The NCSA PKI collects no private information.

\subsection{Compromise and disaster recovery}

The NCSA PKI is housed in NCSA's main machine room and makes use of
NCSA's general disaster recovery procedures.

The NCSA PKI maintains two identical machines with identical Hardware
Security Module to protect against the failure of a single piece of
hardware.

\subsubsection{Due dilligence for subscribe induced compromises}

Software controls will enforce strong passwords on private keys.

%\bibliography{biblio}

\end{document}
